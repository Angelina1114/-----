\documentclass[aspectratio=169]{beamer}
\usetheme{metropolis}
\usepackage{xeCJK}
\setCJKmainfont{Noto Sans TC}
\usepackage{amsmath, amssymb, amsthm}

\title{範疇論與集合論:以 Relation 為例}
\author{盧詠涵\\輔仁大學資訊數學系}
\date{2025年11月12日}

\begin{document}

%------------------------------------------------------
\begin{frame}
  \titlepage
\end{frame}

%------------------------------------------------------
\begin{frame}{研究背景}
\begin{itemize}
  \item 1945 年,Eilenberg 與 Mac Lane 引入「範疇」概念,以形式化數學結構之間的對應關係。
  \item 範疇論提供一種以「關係」而非「元素」為中心的結構性語言。
  \item 本報告以 Relation 為例,展示如何形成範疇。
\end{itemize}
\end{frame}

%------------------------------------------------------
\begin{frame}{為什麼需要抽象化?什麼是抽象化?}
  \begin{itemize}
    \item 不同數學領域在形式上共享相同的結構邏輯。
    \item 抽象化不是讓概念變模糊,而是去除細節、保留本質。
    \item 它讓我們從「元素」轉向「結構」的思考方式。
    \item 範疇論正是這種思想的體現——專注於可合成的結構關係。
  \end{itemize}
  \end{frame}

  %------------------------------------------------------

  \begin{frame}{不同數學對象範疇化所需的性質}
    \begin{itemize}
      \item 各種數學對象若要構成範疇,必須具備能「保持結構」的態射與可合成性。
      \item 不同對象在此基礎上會多出額外的性質:
    \end{itemize}
    
    \begin{table}[h!]
      \centering
      \begin{tabular}{|p{3cm}|p{8cm}|}
      \hline
      \textbf{對象類型} & \textbf{要多符合的條件} \\
      \hline
      關係(Relation) & 只要能表示元素之間的關聯,並能合成就可以。 \\
      \hline
      函數(Function) & 每個輸入只能對應到一個輸出。 \\
      \hline
      群(Group) & 對應要能保持群的運算規則。 \\
      \hline
      拓樸空間(Top) & 對應要能保持空間的連續性。 \\
      \hline
      向量空間(Vect) & 對應要能保持加法和數乘。 \\
      \hline
      \end{tabular}
      \end{table}
    
    \begin{itemize}
      \item Relation 是最簡潔的例子,只依靠「關聯與合成」即可形成範疇。
    \end{itemize}
    \end{frame}
    


%------------------------------------------------------
\begin{frame}{範疇論與集合論的關係}
\begin{itemize}
  \item 集合論:描述元素與成員關係。
  \item 範疇論:描述物件之間的結構與合成規則。
  \item Relation 是範疇論的具體例子之一,不是起點。
\end{itemize}
\end{frame}

%------------------------------------------------------
\begin{frame}{Relation 範疇範例}
  \begin{itemize}
    \item 定義關係:$R \subseteq A \times B$。
    \item 合成:$S \circ R = \{(a,c)\mid \exists b\in B, (a,b)\in R, (b,c)\in S\}$。
    \item 恆等:$I_A = \{(a,a)\mid a\in A\}$。
    \item 這些結構形成範疇 $\mathbf{Rel}$。
  \end{itemize}
  
  \begin{table}[h!]
  \centering
  \begin{tabular}{|p{5cm}|p{6cm}|}
  \hline
  \textbf{範疇論中的概念} & \textbf{在 $\mathbf{Rel}$ 中的對應} \\
  \hline
  物件(Object) & 集合(Set) \\
  \hline
  態射(Morphism) & 關係(Relation)$R \subseteq A \times B$ \\
  \hline
  態射合成(Composition) & 關係的合成 $S \circ R$ \\
  \hline
  恆等態射(Identity Morphism) & 恆等關係 $I_A = \{(a,a)\mid a\in A\}$ \\
  \hline
  範疇結構 & $(\text{集合}, \text{關係}, \circ, I)$ \\
  \hline
  \end{tabular}
  \end{table}
  \end{frame}
  

%------------------------------------------------------
\begin{frame}{C1. 結合律}
\textbf{Axiom C1.} $T\circ(S\circ R) = (T\circ S)\circ R$
\textbf{Proof.}\\

Let $R, S, T$ be relations of $A$ to $B$, $B$ to $C$, and $C$ to $D$.  
Suppose $(a,d)\in T\circ (S\circ R)$ for some $a\in A$ and $d\in D$.
\[
\begin{aligned}
&\Leftrightarrow (\exists c\in C)[(a,c)\in S\circ R \land (c,d)\in T]\\
&\Leftrightarrow (\exists c\in C)[(\exists b\in B)[(a,b)\in R \land (b,c)\in S] \land (c,d)\in T]\\
&\Leftrightarrow (\exists c\in C)(\exists b\in B)[(a,b)\in R \land (b,c)\in S \land (c,d)\in T]\\
&\Leftrightarrow (\exists b\in B)(\exists c\in C)[(b,c)\in S \land (c,d)\in T \land (a,b)\in R]\\
&\Leftrightarrow (\exists b\in B)[(b,d)\in T\circ S \land (a,b)\in R]\\
&\Leftrightarrow (a,d)\in (T\circ S)\circ R.
\end{aligned}
\]
Therefore, $(T\circ S)\circ R = T\circ (S\circ R).$
\end{frame}

%------------------------------------------------------
\begin{frame}{C2. 合成閉合性}
\textbf{Axiom C2.} 若 $S\circ R$ 與 $T\circ S$ 定義,則 $T\circ S\circ R$ 亦定義。\\
\textbf{Proof.}

Let $T\circ S$ and $S\circ R$ be defined.  
\[
\begin{aligned}
S\circ R &= \{(a,c)\mid \exists b\in B, (a,b)\in R \land (b,c)\in S\}\\
&\Rightarrow (\exists c\in C)[(b,c)\in S \land (c,d)\in T]\land (\exists b\in B)[(a,b)\in R \land (b,c)\in S]\\
&\Rightarrow (\exists c\in C)(\exists b\in B)[(a,b)\in R \land (b,c)\in S \land (c,d)\in T]\\
&\Rightarrow (a,d)\in T\circ (S\circ R).
\end{aligned}
\]
Therefore, $T\circ S\circ R$ is defined.
\end{frame}

%------------------------------------------------------
\begin{frame}{C3. 恆等律}
\textbf{Axiom C3.} $R\circ I_A = R = I_B\circ R$\\
\textbf{Proof.} 
There exists an identity relation $I_A$ and $I_B$.
\[
I_A = \{(a,a)\mid a\in A\}, \quad I_B = \{(b,b)\mid b\in B\}.
\]
Let $(a,b)\in R\circ I_A$. \\
Then
\[
(\exists a')[(a,a')\in I_A \land (a',b)\in R].
\]
Since $I_A=\{(a,a)\mid a\in A\}$, we have $a'=a$.  
Hence $(a,b)\in R \Rightarrow R\circ I_A \subseteq R$.

Let $(a,b)\in R$. 
\\Then
\[
(\exists a')[(a',a)\in I_A \land (a,b)\in R] \Rightarrow (a,b)\in R\circ I_A.
\]
\\Thus, $R\subseteq R\circ I_A$. Therefore $R\circ I_A=R$.

Similarly, $I_B\circ R = R$.  
Therefore, $I_B\circ R = R = R\circ I_A.$
\end{frame}

%------------------------------------------------------
\begin{frame}{C4. 恆等映射存在}
\textbf{Axiom C4.} 存在映射 $e:O\to M, e(A)=I_A$,使得
$$R\circ e(A)=R=e(B)\circ R.$$\\
\textbf{Proof.}\\

Let $e(A)=I_A=\{(a,a)\mid a\in A\}$.  \\
By C3, $R\circ e(A)=R$ and $e(B)\circ R=R$.  \\
Therefore, $e:\mathcal{O}\to\mathcal{M}$ is well-defined, and satisfies C4.
\end{frame}

%------------------------------------------------------
\begin{frame}{C5. 恆等唯一性}
\textbf{Axiom C5.} 若 $I_A=I_B$,則 $A=B$。\\
\textbf{Proof.}

\[
I_A=I_B \iff A=B.
\]
$(\Rightarrow)$ Suppose that $I_A=I_B$.  
$I_A=\{(a,a)\mid a\in A\}, \quad I_B=\{(b,b)\mid b\in B\}.$
Let $(a,a)\in I_A$, then $a\in A$.  
Since $I_A=I_B$, we have $(a,a)\in I_B$.  
By the definition of $I_B$, $a\in B$.  
Thus $A\subseteq B$.  

Similarly, take $(b,b)\in I_B \Rightarrow B\subseteq A$.  
Hence $A=B$.

$(\Leftarrow)$ Suppose that $A=B$. Then
$I_A=\{(a,a)\mid a\in A=B\}=\{(a,a)\mid a\in B\}=I_B.$
Thus, $I_A=I_B$.  
Therefore, $I_A=I_B \iff A=B.$
\end{frame}

%------------------------------------------------------
\begin{frame}{結論}
\begin{itemize}
  \item $(O,M,\circ,e)=(\text{所有集合},\text{所有關係},\circ,e)$ 滿足 C1--C5。
  \item 因此,Relation 的系統構成一個範疇 $\mathbf{Rel}$。
  \item 它展示了集合論結構如何成為範疇論的具體例子。
\end{itemize}
\end{frame}

\end{document}
  