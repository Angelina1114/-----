\documentclass[aspectratio=169]{beamer} % 設定比例為 16:9

% --- 中文設定 (必須使用 XeLaTeX 編譯) ---
\usepackage{xeCJK}
\setCJKmainfont{Taipei Sans TC Beta} % 請改為你電腦有的字型,如 "Microsoft JhengHei" (微軟正黑體) 或 "PingFang TC"
\setCJKsansfont{Taipei Sans TC Beta}

% --- 主題設定 ---
\usetheme{Madrid}      % 佈景主題 (Madrid, Antibes, Berlin 等)
\usecolortheme{default} % 顏色主題
\setbeamertemplate{navigation symbols}{} % 隱藏右下角的導航圖示

% --- 資訊設定 ---
\title{這是簡報標題}
\subtitle{副標題或課程名稱}
\author{你的名字}
\institute{學校或機構名稱}
\date{\today}

\begin{document}

% 第一頁:標題頁
\begin{frame}
    \titlepage
\end{frame}

% 第二頁:大綱
\begin{frame}{大綱}
    \tableofcontents
\end{frame}

% --- 第一章節 ---
\section{簡介}

\begin{frame}{這一頁的標題}
    這是普通文字。你可以在這裡輸入介紹內容。
    
    \begin{itemize}
        \item 這是列點 (Itemize)
        \item 這是第二點
        \item 支援數學公式:$E = mc^2$
    \end{itemize}
\end{frame}

% --- 第二章節:區塊與數學 ---
\section{數學與定理}

\begin{frame}{定理與定義區塊}
    Beamer 非常適合展示數學定義。

    \begin{definition}[定義名稱]
        這是一個定義區塊,可以用來解釋重要概念。
    \end{definition}

    \begin{alertblock}{重要提示}
        這是一個紅色的警告區塊,用來強調重點。
    \end{alertblock}

    \begin{exampleblock}{範例}
        這是一個綠色的範例區塊。
        $$ \int_{a}^{b} f(x) \,dx = F(b) - F(a) $$
    \end{exampleblock}
\end{frame}

% --- 第三章節:左右分欄 ---
\section{排版技巧}

\begin{frame}{左右分欄排版}
    \begin{columns}
        % 左欄 (佔寬度 50%)
        \begin{column}{0.5\textwidth}
            這裡是左邊的文字。
            \begin{itemize}
                \item 條列項目 A
                \item 條列項目 B
            \end{itemize}
        \end{column}
        
        % 右欄 (佔寬度 50%)
        \begin{column}{0.5\textwidth}
            % 這裡通常放圖片,例如:
            % \includegraphics[width=\textwidth]{image.jpg}
            這裡是右邊的文字或圖片位置。
        \end{column}
    \end{columns}
\end{frame}

% --- 結尾 ---
\begin{frame}
    \centering
    \Huge 謝謝聆聽!
\end{frame}

\end{document}