\documentclass[12pt]{beamer}

\usetheme{Madrid}
\usecolortheme{default}
\setbeamertemplate{navigation symbols}{}

\title{TCP 擁塞控制與封包行為的數學模型可視化研究}
\author{Your Name}
\date{\today}

\begin{document}

%===========================================================
\begin{frame}
  \titlepage
\end{frame}
%===========================================================

%===========================================================
\begin{frame}{研究主題}
本研究目標是以數學方式重建 TCP 行為,並透過模擬器與圖表可視化顯示:
\begin{itemize}
    \item 三次握手與封包傳輸流程
    \item 擁塞控制(Slow Start / Congestion Avoidance)
    \item 不同網路條件下的 CWND 與 SSTHRESH 成長
    \item 封包遺失(Bernoulli)對 TCP 的影響
    \item GUI 即時可視化與測試後產生圖表
\end{itemize}
\end{frame}
%===========================================================

%===========================================================
\begin{frame}{TCP 的數學形式化:有限狀態機 (FSM)}
\begin{itemize}
    \item TCP 狀態集合:
    \[
    S=\{CLOSED, LISTEN, SYN\_SENT, SYN\_RECEIVED, ESTABLISHED, \dots\}
    \]
    \item 狀態轉移函數:
    \[
    \delta : S \times E \to S
    \]
    \item 可完整分析:
    \begin{itemize}
        \item SYN 丟失
        \item 半開連線 (Half-open)
        \item DDoS 對握手的影響
    \end{itemize}
\end{itemize}
\end{frame}
%===========================================================

%===========================================================
\begin{frame}{擁塞控制的數學模型}
\textbf{Slow Start}:
\[
cwnd = cwnd + 1
\]

\textbf{Congestion Avoidance}:
\[
cwnd = cwnd + \frac{1}{cwnd}
\]

\textbf{Loss 時}:
\[
ssthresh = \frac{cwnd}{2},\quad cwnd \to 1
\]

\begin{itemize}
    \item 使 TCP 在無壅塞時快速成長
    \item 一旦偵測壅塞則降低速率以避免崩潰
\end{itemize}
\end{frame}
%===========================================================

%===========================================================
\begin{frame}{封包遺失的機率模型:Bernoulli 分佈}
每一個封包的遺失建模為:
\[
X \sim Bernoulli(p)
\]

若 $X=1$(封包遺失)則:
\begin{itemize}
    \item cwnd 快速下降
    \item ssthresh 下修
    \item TCP 回到 Slow Start
\end{itemize}

此模型使我們能以數學方式研究攻擊或網路品質下降。
\end{frame}
%===========================================================

%===========================================================
\begin{frame}{模擬器架構}
本研究實作三層式 TCP 模擬器:

\begin{enumerate}
    \item \textbf{Network Layer}
        \begin{itemize}
            \item 延遲 (delay)
            \item 頻寬 (bandwidth)
            \item 封包遺失 (Bernoulli loss)
        \end{itemize}

    \item \textbf{TCP Connection Layer}
        \begin{itemize}
            \item 三次握手
            \item FSM 狀態轉換
            \item cwnd / ssthresh 更新
            \item ACK、重傳邏輯(可擴展)
        \end{itemize}

    \item \textbf{Simulator (DES)}
        \begin{itemize}
            \item 離散事件系統(Future Event List)
            \item 封包排程與到達時間
        \end{itemize}
\end{enumerate}
\end{frame}
%===========================================================

%===========================================================
\begin{frame}{結果展示:無封包遺失}
\begin{itemize}
    \item ssthresh 為水平線:無 loss → 不下降
    \item cwnd 前段快速上升:Slow Start
    \item cwnd 達 ssthresh 後轉為線性成長:Congestion Avoidance
\end{itemize}

此行為與 TCP Reno 模型完全一致,驗證模擬器正確性。
\end{frame}
%===========================================================

%===========================================================
\begin{frame}{結果展示:輕度封包遺失}
\begin{itemize}
    \item cwnd 產生明顯掉落
    \item ssthresh 下修至 cwnd / 2
    \item TCP 重新進入 Slow Start → 再進入 CA
\end{itemize}

可視化出 loss 對 TCP 傳輸效率的直接影響。
\end{itemize}
\end{frame}
%===========================================================

%===========================================================
\begin{frame}{結果展示:延遲與頻寬的影響}
\begin{itemize}
    \item 高延遲 → RTT 增加 → cwnd 成長速度變慢
    \item 低頻寬 → 傳輸時間上升 → ACK 回來變慢
    \item 整體 throughput 明顯下降
\end{itemize}

顯示出網路環境與 TCP 數學行為之間的關聯性。
\end{frame}
%===========================================================

%===========================================================
\begin{frame}{三次握手封包遺失分析}
\begin{itemize}
    \item SYN 丟失 → Client 需重傳(模擬器目前無 RTO)
    \item SYN/ACK 丟失 → Server 重傳(模擬器目前無 RTO)
    \item ACK 丟失 → Server 重傳 SYN/ACK(模擬器目前無 RTO)
\end{itemize}

重傳計時器(RTO)是 TCP 的核心機制,未來可加入模擬器中以強化真實性。
\end{itemize}
\end{frame}
%===========================================================

%===========================================================
\begin{frame}{研究貢獻}
\begin{itemize}
    \item 建立 TCP 擁塞控制的完整數學模型
    \item 實作基於 DES 的 TCP 模擬平台
    \item 以 GUI 即時視覺化三次握手與封包行為
    \item 自動產生 CWND / SSTHRESH 圖表供研究使用
    \item 建立可用於教學、資安分析、性能比較的基礎系統
\end{itemize}
\end{frame}
%===========================================================

%===========================================================
\begin{frame}{研究動機(結尾)}
TCP 是網路世界的基礎協定,但其內部行為難以直接觀察。

本研究希望:
\begin{itemize}
    \item 讓 TCP 不再是黑盒子,而是可分析的數學系統
    \item 以模型精準理解擁塞、攻擊(loss)、環境變化的影響
    \item 建立可視化工具協助教學與資安研究
\end{itemize}

\centering
\textbf{透過數學,我們能真正理解網路。}
\end{frame}
%===========================================================

\end{document}
