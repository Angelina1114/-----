\documentclass[aspectratio=169]{beamer}
\usetheme{metropolis}

%===============================
% 字體設定(Windows)
%===============================
\usepackage{xeCJK}
\usepackage{fontspec}

\setCJKmainfont{Microsoft JhengHei}[
  BoldFont=Microsoft JhengHei Bold,
  AutoFakeSlant=true
]
\setmainfont{Times New Roman}
\setsansfont{Arial}

%===============================
\usepackage{amsmath, amssymb}
\usepackage{graphicx}
\usepackage{tabularx}
\usepackage{enumitem}

%===============================
\title{
{\small 2025 校際應用數學學生研討會} \\
TCP Reno 擁塞控制行為之模擬與視覺化分析
}

\author{報告人:簡偉恆}
\date{2025/12/21(日)}

\begin{document}

%===============================
\maketitle
%===============================

%==========================================================
\begin{frame}{學習背景與目的}
\centering
目標:

\begin{itemize}[label=$\bullet$]
    \item \textbf{依據程榮祥(2001)碩士論文之研究中對 TCP Reno 的描述,重現其核心擁塞控制行為}
    \item 將論文中以數學分析與示意圖呈現的 Reno 機制,轉為可操作的程式模擬流程
    \item 透過模擬觀察 \texttt{cwnd}、\texttt{ssthresh} 與封包遺失事件之動態演化
\end{itemize}

\end{frame}
    

%==========================================================
\begin{frame}{定位}
\centering

\begin{itemize}[label=$\bullet$]
    \item 以 Python 重建 TCP Reno 模型
    \item 視覺化呈現四大階段:SS、CA、FR、Timeout
    \item 驗證行為是否與理論一致
\end{itemize}

\vspace{0.5cm}
著重於重現,不涉新演算法。
\end{frame}

%==========================================================
\begin{frame}{主要成果}
以 Python 建立 TCP 傳輸流程與擁塞控制模擬器,重現 TCP Reno 四大核心行為:

\begin{enumerate}[label=(\arabic*)]
\item \textbf{Slow Start}:\(cwnd\) 指數成長
\item \textbf{Congestion Avoidance}:超過 \(ssthresh\) 後線性成長
\item \textbf{Fast Retransmit}:三次重複 ACK 偵測封包遺失
\item \textbf{Fast Recovery}:\(cwnd\) 部分下降、快速恢復
\end{enumerate}

Reno 行為成功被可視化,有助教學及後續比較 CUBIC、BBR 等演算法
\end{frame}

%==========================================================
\begin{frame}{Slow Start:指數成長}
初始設定: \(cwnd = 1\)。

每收到一個 ACK:

\[
cwnd \gets cwnd + 1
\]

每個 RTT:

\[
cwnd \approx 2^k
\]

呈現指數級膨脹,直到 \(cwnd = ssthresh\) 為止。
\end{frame}

%==========================================================
\begin{frame}{Congestion Avoidance:線性成長}
當 \(cwnd \ge ssthresh\):

\[
cwnd \gets cwnd + \frac{1}{cwnd}
\]

每個 RTT 約成長 \(+1\) MSS。

呈現穩定、線性的成長。
\end{frame}

%==========================================================
\begin{frame}{Fast Retransmit 與 Fast Recovery}
若收到 \textbf{三次重複 ACK}:

\[
ssthresh = \frac{cwnd}{2},\qquad cwnd = ssthresh
\]

若發生 timeout:

\[
cwnd = 1,\qquad ssthresh = \frac{ssthresh}{2}
\]

Timeout 的懲罰比 Fast Retransmit 更重。
\end{frame}

%==========================================================
\begin{frame}{模擬流程架構}
實作 TCP 的核心流程:

\begin{itemize}[label=$\bullet$]
\item 三次握手、封包傳輸、ACK 機制
\item 可調整延遲(latency)、丟包率(loss)、頻寬(bandwidth)
\item 即時顯示:\(cwnd\)、\(ssthresh\)、封包事件
\end{itemize}

\begin{center}
\begin{tabularx}{0.95\textwidth}{|c|X|}
\hline
模組 & 功能描述 \\
\hline
Connection & SEQ、ACK、封包收發 \\
\hline
Simulator & 時間軸事件、延遲、丟包 \\
\hline
Reno Engine & Slow Start、CA、Fast Retransmit、Timeout \\
\hline
GUI & 即時曲線視覺化 \\
\hline
\end{tabularx}
\end{center}
\end{frame}

%==========================================================
\begin{frame}{無丟包情況}
\begin{center}
\includegraphics[width=0.8\textwidth]{img/reno0_pic.png}
\end{center}
\end{frame}

%==========================================================
\begin{frame}{有丟包情況(10\%)}
\begin{center}
\includegraphics[width=0.8\textwidth]{img/reno_10.png}
\end{center}

\end{frame}

%==========================================================
\begin{frame}{數學觀點:離散動力系統}
TCP Reno 可視為狀態轉換系統:

\[
S \xrightarrow{f} S'
\]

其中 \(f\) 為封包事件(ACK/loss/timeout)。

其結構對應:

\begin{itemize}[label=$\bullet$]
\item \(cwnd\):控制變數
\item \(ssthresh\):分段條件(piecewise condition)
\item duplicate ACK:局部 reset
\item timeout:全域 reset
\end{itemize}

呈現與離散動力系統相近的遞迴行為。
\end{frame}

%==========================================================
\begin{frame}{動機}
\begin{itemize}[label=$\bullet$]
\item 多數教材以靜態圖示呈現 TCP,難以理解動態邏輯
\item 希望重建真實的 cwnd 與 ssthresh 時間演化
\item 探索「數學模型」與「網路行為」的結構相似性
\item 建立未來比較 CUBIC、BBR 等演算法的平台
\end{itemize}
\end{frame}

%==========================================================
\begin{frame}{總結}
\centering
\begin{enumerate}[label=$\bullet$]
\item 成功以模擬器重現 TCP Reno 的動態行為
\item 可即時觀察 \(cwnd\)、\(ssthresh\)、loss 時間點
\item 模型與理論一致
\item 可延伸至 CUBIC / BBR 等現代演算法研究
\end{enumerate}
\end{frame}

%==========================================================
\begin{frame}{References}
    \footnotesize
    
    [1]  程榮祥(2001)。《TCP 擁塞控制之效能改進》。  
    國立成功大學資訊工程研究所碩士論文。
    
    \vspace{0.8em}
    
    [2]Jacobson, V. (1988). \textit{Congestion Avoidance and Control}. 
    ACM SIGCOMM Computer Communication Review.
    
    \vspace{0.8em}
    
    [3] Jerry Banks、John S. Carson II、Barry L. Nelson、David M. Nicol 著,  
    肖田元、范文慧 譯(2008)。《離散事件系統仿真(原書第 4 版)》。  
    北京:機械工業出版社。
    
    \vspace{0.8em}
    
    [4] Postel, J. (1981). \textit{RFC 793: Transmission Control Protocol}.  
    Internet Engineering Task Force (IETF).
    
\end{frame}

%==========================================================
\end{document}
